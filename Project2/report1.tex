
\def\problemset#1#2{
\begin{center}
\framebox{\framebox{\vspace{0.5in}
  \parbox{12cm}{\bf
    Gaurav Nanda \hfill Homework \#2\\
    Natural Language Processing \hfill #2, 2014}
  }}\vspace{0.05in}
\end{center}
}

\documentclass[11pt]{article}
\usepackage{fullpage}
\usepackage{algorithm}
\usepackage{algpseudocode}
\usepackage{hyperref}
\usepackage{footnote}

\setlength{\parindent}{0in}
\setlength{\parskip}{2mm}

%\setlength{\oddsidemargin}{-0.5cm}
%\setlength{\evensidemargin}{-0.5cm}
%\setlength{\textwidth}{18cm}

%\setlength{\topmargin}{-1.7cm}
%\setlength{\textheight}{25cm}
\begin{document}
\problemset{1}{March 1}

\section{Problem Statement}

There exist probabilistic sequence models that allow integrating uncertainty over multiple, interdependent classifications and collectively determine the most likely global assignment of Part of Speech Tags (POS). Two such standard models are Hidden Markov Model  (HMM) and Conditional Random Field (CRF).

Our task is to evaluate HMM and CRF models on various parameters including training accuracy, testing accuracy, runtime, accurate tag estimation for Out of Vocabolary(OOV) Words and addition of orthographic features. Both of these models are to be evaluated on ATIS ans WSJ data sets from the Penn Treebank.

\section {Testing Setup}

\subsection {Mallet Format Input}
Two models have been compared using Mallet's CRF and HMM implementations. To convert Pen TreeBank's raw data to Mallet format, I have implemented RawToMallet.java. For adding orthographic features to tagged output file, one can pass a feature file as input with suitable flags to include or exclude a particular feature.
\begin{verbatim}
~/:>cat features
CAPS  true
SUFFIX false
PREFIX true
HYPHEN false
START_NUMBER false

~/:>java RawToMallet atis3.pos feature_mallet_files/atis3_caps_prefix.mallet ./features
\end{verbatim}

\subsection {Handling OOV Tokens}

To estimate the count and testing accuracy of OOV Tokens, "evaluateInstanceList" function in "TokenAccuracyEvaluator.java" has been modified. During first iteration, all the training tokens are stored in a HashMap and all the test tokens are looked up in the Map to estimate the count and accuracy the OOV tokens.

\section {Experiments}

Various experiments have been performed to compare two models on various attributes. All the numbers computed for ATIS have been averaged over ten random seeds. ATIS is using 80\% of data for training and WSJ is using 50\% data for training. 

\subsection {Overall Test Accuracy}

\begin{center}	
	\begin{table}[ht]
  	\centering
   	\begin{tabular}{| l | l | l | l | l | l |}
    	\hline
        Model & HMM & CRF & Total TokenCount \\ \hline
        ATIS & 0.8608 & 0.9216 &  3283 \\ \hline
        WSJ (00 and 01) & 0.7859 & 0.8066 & 73576 \\ \hline
	WSJ (00, 01, 02 and 03) & 0.8324 & 0.8423 & 155513 \\ \hline
    	\end{tabular}
    	\caption{Overall Test Accuracy HMM vs CRF. }
    	\end{table}%
\end{center}

{\bfseries Analysis}

1. CRF model performs better than HMM model.

In general, generative models have to make strict independence assumptions on observations to achive tractability which reduces performance. For HMM model, this independence assumption is relaxed by arranging the output variables in a linear chain. But still to come up with a tractable model, it assumes that  state depends only on its immediate predecessor, that is, each state y$_t$ is independent of all its ancestors y$_1$, y$_2$, . . . , y$_{t-2}$ given its previous state y$_{t-1}$. Second, an HMM assumes that each observation variable x$_t$ depends only on the current state y$_t$. CRF does not make such assumptions and is a conditional probabilistic model.

Therefore, because of these assumptions, HMM does not perform as good as CRF model.

2. For WSJ corpus, testing accuracy keeps on increasing as we keep on increasing token count. With the addition of more data, models are better trained and perform better for test data. 

3. In the case of ATIS, data is very less diverese. Therefore, testing accuracy is high for both HMM and CRF. Even if we use only 20\% of the data in training, we still get accuracy as high as 0.863.

\subsection {Test accuracy for OOV items}

\begin{center}	
	\begin{table}[ht]
  	\centering
   	\begin{tabular}{| l | l | l | l | l | l | l |}
    	\hline
        Model & HMM & CRF & OOV TC & Testing TC \\ \hline
        ATIS & 0.2680 & 0.3058 &  25 & 641 \\ \hline
        WSJ (00 and 01) & 0.3819 & 0.4760 & 5735 & 37465 \\ \hline
	WSJ (00, 01, 02 and 03) & 0.3938 & 0.5017 & 9340 & 81939 \\ \hline
    	\end{tabular}
    	\caption{OOV Test Accuracy HMM vs CRF. }
    	\end{table}%
\end{center}

{\bfseries Analysis}

As expected, test accuracy for OOV is very less in comparison to the seen tokens. Because these tokens were not present during training, their observation probabilities would be very less (primarily, dependent upon smoothing techniques used). However, CRF being a discriminative model still performs better than the HMM.

\subsection {Training Accuracy}

\begin{center}	
	\begin{table}[ht]
  	\centering
   	\begin{tabular}{| l | l | l | l | l | l |}
    	\hline
        Model & HMM & CRF & Total TokenCount \\ \hline
        ATIS & 0.8886 & 0.9990 &  3283 \\ \hline
        WSJ (00 and 01) & 0.8627 & 0.9953 & 73576 \\ \hline
	WSJ (00, 01, 02 and 03) & 0.8877 & 0.9936 & 155513 \\ \hline
    	\end{tabular}
    	\caption{Training Accuracy HMM vs CRF. }
    	\end{table}%
\end{center}

Training accuracy for CRF is much higher than HMM models. This is expected because CRF model uses L-BFGS optimization procedure to maximize the conditional log likelihood of the supervised training data. At the same time, CRF is more likely to overfit the training data also. This is the reason that training accuracy for ATIS is very high.

\end{document}